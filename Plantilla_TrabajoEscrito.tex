%%%%%%%%%%%%%%%%%%%%%%%%%%%%%%%%%%%%%%%%%%%%%%%%%%%%%%%%%%%%%%%%%%%%%%%%%%%%%%%%%%%%%%%%%%%%%%%
%Plantilla: para la realizaci�n de informes.
%Curso:     Simulaci�n estad�stica.
%Profesor:  Johann A. Ospina.
%%%%%%%%%%%%%%%%%%%%%%%%%%%%%%%%%%%%%%%%%%%%%%%%%%%%%%%%%%%%%%%%%%%%%%%%%%%%%%%%%%%%%%%%%%%%%%%


%Establece el tipo de documento (art�culo), tama�o de letra (10pt) a una columna.
\documentclass[letterpaper,12pt,onecolumn,titlepage]{article} 
 
 
% Cargar paquetes
\usepackage{verbatim}
\usepackage{mathrsfs}
\usepackage{amsmath}
\usepackage{amssymb}
\usepackage{subfigure}
\usepackage{ucs}
\usepackage[latin1]{inputenc}
\usepackage[spanish]{babel}
\usepackage{fontenc}
\usepackage{graphicx}
\usepackage{anysize}
\usepackage{fancyhdr}
\usepackage[comma,authoryear]{natbib}
\usepackage{url} %paquete para definir url
\usepackage{hyperref}  %hipervinculos

%Estilo de la p�gina
\pagestyle{fancy}

%Establecer el margen
\marginsize{2cm}{2cm}{1cm}{1cm}
\setlength{\headheight}{13.1pt}


% Portada
\title{
    \textbf{Introducci�n al Equating}\
    ~\\{Simulaci�n Estad�stica}   
    }
\author{
    {Kevin Steven Garcia Chica Cod. 1533173}
 ~\\{Cesar Andres Saavedra Vanegas Cod. 1628466}}






\date{
     \textbf{Universidad Del Valle}\   
    ~\\{Facultad De Ingenieria}
    ~\\{Estadistica}
    ~\\{Febrero}
    ~\\{2018}}
 
 
 
\decimalpoint %Poner punto decimal
 
\begin{document}
 
% Se aplica el formato a las p�ginas. Se despliegan: portada e �ndices de materias, figuras y tablas
\renewcommand{\listtablename}{}
\renewcommand{\tablename}{Tabla}
\maketitle
\setcounter{page}{2}
\tableofcontents{}
%\thispagestyle{empty}
%\newpage
\listoffigures{}
\listoftables{}

\thispagestyle{empty}

\newpage
\fancyhead{}
\fancyfoot{}
 
% Encabezado y pie de pagina
\lhead{Simulaci�n Estad�stica}
\lfoot{Universidad Del Valle}
\rfoot{\thepage}

% Estilo de la bibliograf�a
\bibliographystyle{apalike}
 
% Desarrollo de los contenidos del documento
\begin{center}
\textbf{\title{EQUATING}}
\end{center}


~\\ Equating (equiparando) de forma general, es un proceso estad�stico cuyo objetivo o finalidad es ajustar los puntajes de dos formas distintas de una misma prueba; con esto, se busca relacionar el puntaje de una forma de una prueba y su equivalente en la otra forma con la cual se quiere comparar o equiparar, en otras palabras, lo que se busca aplicando Equating es que los puntajes en los formularios de prueba se puedan usar indistintamente.

~\\ Debemos tener en cuenta que condiciones o supuestos se debe cumplir para poder aplicar Equating. Para comparar dos formas diferentes de un test se deben cumplir b�sicamente los siguientes 5 supuestos:
~\begin{itemize}
\item Simetr�a:
\item Igual o cercana confiabilidad:
\item Equidad:
\item Invarianza poblacional:
\item Igual constructo: Esta condici�n nos dice que ambas formas del test deben medir el mismo constructo o las mismas caracter�sticas.
\end{itemize}

~\\ En general, las situaciones en las que se requiere el uso de Equating son en la aplicaci�n de distintas formas de una misma prueba o test. Por ejemplo, en un examen de ingreso a estudios superiores en el que se convoca a los aspirantes para distintas fechas resulta extremadamente conveniente disponer de formas alternativas de la prueba o del examen, por razones estrictamente de seguridad (evitar plagio o conocimiento previo de la prueba). Tambi�n es necesario disponer de distintas formas de un test cuando se desea medir en repetidas ocasiones a un mismo individuo o colectivo con el fin de evaluar, por ejemplo, su progreso acad�mico o un posible cambio en sus actitudes. En cualquiera de estos casos, para poder comparar las puntuaciones obtenidas en las distintas formas del test es necesario ponerlas previamente en la misma escala, y eso lo logramos mediante el uso adecuado del Equating.

\begin{center}
\pagebreak \textbf{\title{Referencias}}
\end{center}

~\begin{itemize}
\item In�s Mar�a Varas C�ceres, 2018. Taller: Introducci�n al Equating. 
\item Fang Chen, Xiaomin Huang y David MacGregor, 2009. EQUATING OR LINKING: BASIC CONCEPTS AND A CASE STUDY.
\item Lady Catheryne Lancheros Florian, 2013, Universidad Nacional de Colombia. M�TODOS DE EQUIPARACI�N DE PUNTUACIONES: LOS EX�MENES DE
ESTADO EN POBLACI�N CON Y SIN LIMITACI�N VISUAL.
\item Neil J. Dorans, Tim P. Moses, and Daniel R. Eignor, 2010. Principles and Practices of Test Score Equating.
\item Navas, M. J. ,2000. Equiparaci�n de puntuaciones: Exigencias actuales y retos de cara al futuro.  

\end{itemize}
\end{document}








